\documentclass[14pt,ngerman]{extarticle}  %% document requires XeLaTex.
\usepackage[ngerman]{babel}
\usepackage{geometry}
 \geometry{
 a4paper,
 total={170mm,257mm},
 left=20mm,
 top=20mm,
}
\usepackage{c-base-cience-ring}


\title{
    {c-base-cience-ring.sty} \\
    Makros für Papers im\\ \cevain{c-base cience ring}
    }

\author{
    \resizebox{8cm}{!}{\documentclass{standalone}
\usepackage{tikz}
\usepackage{calc}
\usepackage{xcolor}
\begin{document}
%% c-base logo nachgebaut von penta.
%% alles nur geschätze Winkel und Abstände :/
%% um den code zu verstehen, einfach mal einzelne Teile auskommentieren und wieder einkommentieren (ctrl-#) und dann mal \draw[white] durch \draw[red] ersetzen, dann sieht man, was was ist.
%% viel Spaß damit.
\tikzset{
  pics/carc/.style args={#1:#2:#3}{
    code={
      \draw[pic actions] (#1:#3) arc(#1:#2:#3);
    }
  }
}
\definecolor{eins}{HTML}{e7e7e8}   %% Ring 1 "core"     - weiß - Mittelpunkt, Ring um Mittelpunkt
\definecolor{zwei}{HTML}{ed1c24}   %% Ring 2 "com"      - rot -  "Fenster" innen
\definecolor{drei}{HTML}{fbad18}   %% Ring 3  "culture" - orange - fünf Module, quasi invertiert
\definecolor{vier}{HTML}{74c043}   %% Ring 4  "creativ" - grün -  vier Module
\definecolor{fuenf}{HTML}{0089d0}  %% Ring 5 "cience"  - cyan (blau) - drei Module mit "Strich"
\definecolor{sechs}{HTML}{11357e}  %% Ring 6 "carbon" -  indigo - viele "Fenster" außen
\definecolor{sieben}{HTML}{000000} %% Ring 7 "clamp" -  schwarz, c-förmig
\definecolor{cbase}{HTML}{222222}  %% Körper der Raumstation


    \begin{tikzpicture}%
        \def\radi{10}%     

        %% äußerer Radius
        \filldraw[sieben] (0:0) circle (\radi-0.3);  %% CLAMP - schwarz

        \filldraw[cbase] (0:0) circle (\radi-1.52);
        
        % was wegnehmen / weiß übermalen
        \draw[white, line width=1.36cm] (0:0) pic{carc=-55:90:\radi-0.9}; 

        %% clamp - viele kleine Fenster
        \foreach [count=\i] \ii in {%
        70,74,78,100,
        114,118,...,170,
        182,186,
        202,206,...,250,
        266,270,...,304}
            \draw[sechs, line width=0.5cm] (0:0) pic{carc=\ii:\ii-2:\radi-2}; 
        
        \draw[white, line width=3.05cm] (0:0) pic{carc=50:-55:\radi-3};
        
        \draw[white, line width=2cm] (0:0) pic{carc=-90:0:\radi-3.5};

  
        \draw[fuenf, line width=1.8cm] (0:0) 
            pic{carc=82:135:\radi-3.5};
        \draw[fuenf, line width=1.8cm] (0:0) 
            pic{carc=65:74:\radi-3.5};
        \draw[fuenf, line width=1.8cm] (0:0) 
            pic{carc=53:62:\radi-3.5};
            
        \draw[cbase, line width=0.4cm] (0:0) 
            pic{carc=110:135:\radi-3.3};

        \foreach [count=\i] \ii in {1,2,3,4}
            \draw[vier, line width=1.5cm] (0:0) 
            pic{carc=90*\i:90*\i+45:\radi-5.5};


        \filldraw[drei] (0:0) circle (\radi-6.5);
        
        \foreach [count=\i] \ii in {1,2,3,4,5}
            \draw[cbase, line width=0.7cm] (0:0) 
            pic{carc=72*\i+27:72*\i+36+27:\radi-7};

        \filldraw[cbase] (0:0) circle (\radi-7.5);

        \foreach [count=\i] \ii in {160,172,...,490}
            \draw[zwei, line width=0.4cm] (0:0) pic{carc=\ii:\ii-8:\radi-8};   
        
        \filldraw[eins] (0:0) circle (\radi-8.5);
        \filldraw[black] (0:0) circle (\radi-8.8);
        \filldraw[eins] (0:0) circle (\radi-9.8);
     
    \end{tikzpicture}
\end{document}
}\\\smallskip\\\cevain{ccr}\\
    % \ceva{c-base cience ring}\\ 
    c-base cience ring\\\smallskip\\
    \cevain{penta}\\ \makeatletter {\texttt{penta@c-base.org}}\makeatother
    }


\date{27. Mai 2024}

\begin{document}

\maketitle

\setlength{\parskip}{2ex}
\setlength{\parindent}{0ex}
 
\section{Requirements}

Package must be used with \XeLaTeX\ and the fonts \verb|ceva-c2.ttf| and \verb|linek.ttf| must be present. You also need

\verb|\documentclass{extarticle}| or \verb|\documentclass{extreport}|. 

Also, for correct display of \linekin{linek}, you must have a file calle \verb|latexmkrc| without file extension (!) that contains the line:

\verb|$xdvipdfmx = "xdvipdfmx -E -o %D %O %S";|

Once this is all provided, load \verb|\usepackage{c-base-cience-ring}| and use the commands described below.

This package loads many useful packages, such as \verb|booktab|, \verb|tikz| and many others. You may want to look at the code for details. You may also want to set up \verb|\hypersetup| for your PDF metadata.

To create a bibliography, the package loads

\verb|\usepackage{biblatex}| and \verb|\addbibresource{literature.bib}|

So please upload / name your bibliography accordingly.

\section{\cevain{ceva}}


\verb|\cevain{foo}| prints ceva and latin: \cevain{foo} and also creates an index entry.

\verb|\ceva{foo}| prints only \ceva{foo} and does not provide an index entry.

The color of \ceva{ceva} can be defined by

\verb|\definecolor{cevacol}{HTML}{000000}| 

The standard is black.

\verb|\cevapic{foo}| should be used within TikZ pictures, since they may not have a defined color \verb|cevacol|.

\verb|\twofonts{my lengthy quote}| typesets longer paragraphs in both \ceva{ceva} and \emph{cursive}. Optional parameter is the source: \verb|\twofonts[\cite{mybibentry}]{foo}| Note that \verb|cite| can then not have an opitonal argument for a page number (sadly), so if you need that, you must include that information (if needed) directly in the optional argument. 

\twofonts[S.1~\cite{adams1981restaurant}]{In the beginning the Universe was created. This has made a lot of people very angry and been widely regarded as a bad move.}

Vertical mode is provided by \verb|\cevavert[opt]{foo}|

\cevavert[option]{be future com pat ible}

which also provides a quote environment.

\section{\linekin{Linek}}

Similarly to \verb|\twofonts|, use \verb|\linek{foo}|, which spells in a quote environment three typefaces:
\linek[citation here!]{foo} 
while

\verb|\linekin{foo}| produces \linekin{foo}

and

\verb|\linekonly{linek}| which produces only 
\linekonly{linek}.

Finally, you can quote \linekonly{linek} in vertical mode with \verb|\linekvert[option]{foo}|:
    \linekvert[option]{be future com pat ible}


\section{Using this for slides}

You can use this for \verb|beamer|, but then you must get rid of the \verb|makeidx|-package sice \verb|beamer| does not support indexation. I have knocked together a stylesheet \verb|c-base-slides.sty| which should do the job. Holler me for more tipps.

\section{Colours}

I define colours in the way of the \cevain{abtöner}. They are stored in \verb|eins|, \verb|zwei| and so forth for all the rings (German numerals for the \cevain{rings}). Use this, or override or change if you are a \cevain{purist}. {\color{zwei} This is colour zwei}, while {\color{drei} this is color drei}.

\begingroup %% Ringe

    %% In this group I had the chapters on the rings, so the section was renamed to ring etc.; I don't think you need this, but if you want, you can of course use this.
    
    \renewcommand{\thesection}{\arabic{section}.~Ring}

    \newcommand{\ringcolor}[1]{
        \begin{tikzpicture}
        \filldraw[fill=#1]  (0,0) rectangle (5ex,1ex);
        \end{tikzpicture}
    }
    
    \newcommand{\csection}[2][.]{
        \section{\hspace{2ex}#2\hspace{2ex} 
            {\ceva{(#2)}}
        }
       \index{{\fontspec{[ceva-c2.ttf]}#2}$\mid$ #2}
        \Hrule[#1]
        \fancyhead[RO]{\thesection~
            \ringcolor{#1}~#2~\ceva{#2}
        }
    }


   
\endgroup
\clearpage

% \renewcommand\indexname{Fachindex}
\indexprologue{\noindent Index entries have been created whenever \texttt{cevain}  or \texttt{linekin} was used.}

\fancyhead[RO]{Fachindex}\addcontentsline{toc}{chapter}{Fachindex}\label{sec:index}
\printindex

\clearpage

  
\fancyhead[LO]{Literatur}\label{sec:literatur}
\fancyhead[RO]{Literatur}\addcontentsline{toc}{chapter}{Literatur}
\printbibliography

 

\end{document}
